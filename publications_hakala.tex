%\documentstyle[12pt,a4]{report}
\documentclass[12pt,a4]{report}
\usepackage{hyperref}

\newcommand{\baselinestrech}{1.2}
\pagestyle{plain}
%
% sivun koon maarittelyt
%
%\setlength{\textwidth}{150mm}
\setlength{\textwidth}{165mm}
%\setlength{\topmargin}{10mm}
\setlength{\topmargin}{1mm}
\setlength{\headheight}{0mm}
\setlength{\headsep}{0mm}
%\setlength{\textheight}{260mm}
\setlength{\textheight}{240mm}
%\setlength{\oddsidemargin}{5mm}
\setlength{\oddsidemargin}{1mm}
%\setlength{\parindent}{0mm}
%\setlength{\parskip}{1ex plus 0.5ex minus 0.2ex}

\setlength{\hoffset}{-1.0cm}

\begin{document}

\newcommand{\emp}[1]
{\it{#1}\rm}

\newenvironment{Body} % Definition of Body
 {\begin{list}{}{%
%	\setlength{\rightmargin}{\rightmargin0cm}}
	\vspace{-0.6cm}
	\setlength{\leftmargin}{1.2cm}}
	  \item[]\ignorespaces}
% {\unskip\end{list}}
 {\end{list}}

%\vspace*{\stretch{1}}

\vspace{-2cm}

%\hfill LIITE 2 (2)

%\begin{center}
%{\bf JULKAISULUETTELO}
\noindent {\bf MIKKO HAKALA} \\

%\noindent {\bf JULKAISULUETTELO} \hfill 25.5.2008\\
\noindent {\bf PUBLICATIONS} \hfill 17.8.2016\\

%
%%\noindent{\it {\bf Double asterisk (**) is marked on the publications which are closely related to the applied three year funding from the University of Helsinki.}}
%\noindent{\bf Double asterisk (**) and boldface} mark the ten most important publications (see separate document for details).
\noindent{\bf Double asterisk (**) and boldface} mark the ten most important publications.
\vspace{0.3cm}

%related to the applied funding.

%

\vspace{0.3cm}


%\end{center}

%\noindent {\bf A ERILLISTEOKSET}\\

%V\"ait\"oskirja: {\it Defect Complexes in Silicon: Electronic Structures and Positron\\ 
%\indent Annihilation}, Teknillinen korkeakoulu, Espoo, 2001\\

\vspace{0.4cm}

%\noindent {\bf B ARTIKKELIT}\\

%%%\noindent B1 Artikkelit tieteellisiss\"a aikakauslehdiss\"a
%%\noindent {\bf Articles in Refereed Journals }
%%%\noindent {\bf ARTICLES IN REFEREED JOURNALS}
%\noindent {\bf Artikkelit tieteellisiss\"a aikakauslehdiss\"a}
%\noindent {\bf Articles in refereed journals}
%\noindent {\bf Peer-reviewed scientific articles}
%\noindent {\bf A Peer-reviewed scientific articles}
\noindent {\bf Peer-reviewed scientific articles (68 publications)}\\

%%{\it{\bf Publications from the last five years: Nos: 32 - 57 (26 publications)}}

\begin{enumerate}

\item \emp{ {Theoretical and experimental study of positron annihilation
with core electrons in solids}},\\M. Alatalo, B. Barbiellini,
M. Hakala, H. Kauppinen, T. Korhonen, M. J. Puska, K. Saarinen,
P. Hautoj\"arvi, and R. M. Nieminen, Phys. Rev. B. {\bf 54}, 2397
(1996).

\item \emp{Correlation effects for electron-positron momentum density
in solids}, \\B. Barbiellini, M. Hakala, M. J. Puska, K. Saarinen,
R. M. Nieminen, and A. A. Manuel, Phys. Rev. B. {\bf 56}, 7136 (1997).

\item \emp{Correlation effects for positron annihilation with core and semicore
electrons},\\ B. Barbiellini, M. J. Puska, M. Alatalo, M. Hakala,
A. Harju, T. Korhonen, S. Siljam\"aki, T. Torsti, R. M. Nieminen,
 Appl. Surf. Sci. {\bf 116}, 283 (1997).

\item \emp{{\bf ** Momentum distributions of electron-positron pairs
 annihilating at vacancy clusters in Si}}, \\M. Hakala, M. J. Puska,
 and R. M. Nieminen, Phys. Rev. B. {\bf 57}, 7621 (1998).

\item \emp{Microscopic identification of native donor Ga-vacancy
complexes in Te-doped GaAs},\\J. Gebauer, M. Lausmann, T. E. M. Staab,
R. Krause-Rehberg, M. Hakala, and M. J. Puska, Phys. Rev. B. {\bf 60},
1464 (1999).

\item \emp{Identification of vacancy-impurity complexes in highly
n-type Si}, \\K. Saarinen, J. Nissil\"a, H. Kauppinen, M. Hakala,
M. J. Puska, P. Hautoj\"arvi, and C. Corbel, Phys. Rev. Lett. {\bf
82}, 1883 (1999).

\item \emp{The structure of
vacancy-impurity complexes in highly n-type Si},\\ K. Saarinen, J. Nissil\"a, H. Kauppinen, M. Hakala,
M. J. Puska, P. Hautoj\"arvi, and C. Corbel,  Physica B {\bf
273-274}, 463 (1999). 

\item \emp{Observation of Ga vacancies and
negative ions in undoped and Mg-doped GaN bulk crystals},\\K. Saarinen, J. Nissil\"a, J. Oila, V. Ranki, M. Hakala,
M. J. Puska, P. Hautoj\"arvi, J. Likonen, T. Suski, I. Grzegory,
B. Lucznik, and S. Porowksi, Physica B
{\bf 273-274}, 33 (1999). 

\item \emp{{Theoretical studies of interstitial boron defects in
silicon}},\\ M. Hakala, M. J. Puska, R. M. Nieminen, Physica B {\bf
273-274}, 268 (1999).

\item \emp{{{\bf ** First-principles calculations of interstitial boron in
silicon}}},\\ M. Hakala, M. J. Puska, and R. M. Nieminen, Phys. Rev. B
{\bf 61}, 8155 (2000).

\item \emp{Irradiation experiment revisited - Stability and positron
lifetime of large vacancy clusters in silicon},\\ T. E. M. Staab,
M. J. Puska, M. Hakala, A. Sieck, M. Haugk, T. Frauenheim, and
H. S. Leipner, Mater. Sci. Forum {\bf 363-365}, 135 (2001).

\item\emp{Native defects and self-diffusion in GaSb},\\ M. Hakala,
M. J. Puska and R. M. Nieminen, J. Appl. Phys. {\bf 91}, 4988 (2002).

\item\emp{Scattering effects in a positron lifetime beam line},\\
A. Laakso, M. O. Hakala, A. Pelli, K. Ryts\"ol\"a, and K. Saarinen,
Mater. Sci. Forum. {\bf 445}, 489 (2004).

\item\emp{{Compton profiles for water and mixed water-neon clusters: A
measure of coordination}},\\ M. Hakala, S. Huotari,
K. H\"am\"al\"ainen, S. Manninen, Ph. Wernet, A. Nilsson, and
L. G. M. Pettersson, Phys. Rev. B {\bf 70}, 125413 (2004).

\item\emp{Electron emission from solids under electron irradiation: a
Monte Carlo study},\\ M. Hakala, C. Corbel and R. M. Nieminen,
J. Phys. D {\bf 38}, 711 (2005).

\item\emp{{{Calculation of valence electron momentum densities using the
projector aug\-mented-wave method}}},\\ I. Makkonen, M. Hakala, and
M. J. Puska, J. Phys. Chem. Solids {\bf 66}, 1128 (2005).

\item\emp{Modeling the momentum distributions of annihilating 
electron-positron pairs in solids},\\ I. Makkonen, M. Hakala, and
M. J. Puska, Phys. Rev. B {\bf 73}, 035103 (2006).

\item\emp{{{Intra- and intermolecular effects in the Compton profile of
water}}},\\ M. Hakala, K. Nyg\aa rd, S. Manninen, L. G. M. Pettersson
and K. H\"am\"al\"ainen, 
Phys. Rev. B {\bf 73} 035432 (2006).

\item\emp{{{Ion hydration studied by X-ray Compton scattering}}},\\
K. Nyg\aa rd, M. Hakala, S. Manninen, K. H\"am\"al\"ainen, M. Itou,
A. Andrejczuk, and Y. Sakurai, 
Phys. Rev. B {\bf 73} 024208 (2006).

\item\emp{First-principles calculation of positron states
and annihilation at defects in semiconductors},\\ I. Makkonen, M. Hakala, and
M. J. Puska, Physica B {\bf 376-377}, 971 (2006). % -974

\item\emp{{{Electronic stucture of methane hydrate studied by Compton scattering}}},\\ C. Sternemann,
S. Huotari, M. Hakala, M. Paulus,
M. Volmer, C. Gutt, T. Buslaps, N. Hiraoka, D. D. Klug, 
K. H{\"a}m{\"a}l{\"a}inen, M. Tolan, and J. S. Tse, 
Phys. Rev. B {\bf 73}, 195104 (2006). % -1-6

\item\emp{{\bf ** {Gold as intermolecular glue: a predicted planar
triaurotriazine, C$_3$Au$_3$N$_3$, isomer of gold cyanide}}},\\
M. O. Hakala and P. Pyykk\"o, Chem. Commun., 2890 (2006). 

%advanced article, published
%on the web 1st June 2006, DOI: 10.1039/b605757h (2006).

\item\emp{{Correlation of hydrogen bond lengths and angles in liquid
water based on Compton scattering}},\\ M. Hakala, K. Nyg\aa rd,
S. Manninen, S. Huotari, T. Buslaps, A. Nilsson, L. G. M. Pettersson,
and K. H\"am\"al\"ainen,  J. Chem. Phys. {\bf 125}, 084504 (2006). % -1-7

\item\emp{Compton scattering study of water versus ice Ih: Intra- and intermolecular structure},\\ K. Nyg\aa rd, M. Hakala, S. Manninen,
A. Andrejczuk, M. Itou, Y. Sakurai, L. G. M. Pettersson
and K. H\"am\"al\"ainen,  
Phys. Rev. E {\bf 74}, 031503 (2006).

\item\emp{{Isotope quantum effects in the Compton profile of water}},\\ K. Nyg\aa rd, M. Hakala, T. Pylkk{\"a}nen, S. Mannine, T. Buslaps, M. Itou, 
A. Andrejczuk, Y. Sakurai, M. Odelius and K. H\"am\"al\"ainen,  
J. Chem. Phys. {\bf 126}, 154508 (2007).

\item\emp{Gold as intermolecular glue: a theoretical study of nanostrips based on quinoline-type monomers},\\ 
P. Pyykk\"o, M. O. Hakala and P. Zaleski-Ejgierd, Phys. Chem. Chem. Phys. {\bf 9}, 3025 (2007).

\item\emp{{{Comparison of chain versus sheet crystal structures for the cyanides MCN (M=Cu-Au) and dicarbides 
MC$_2$ (M=Be-Ba,Zn-Hg)}}},\\P. Zaleski-Ejgierd, M. Hakala, and P. Pyykk\"o, 
Phys. Rev. B {\bf 76}, 094104 (2007). % -1-9

\item\emp{{{Configurational energetics in ice Ih probed by Compton scattering}}},\\ K. Nyg\aa rd, M. Hakala, S. Manninen,
M. Itou, Y. Sakurai and K. H\"am\"al\"ainen,  
Phys. Rev. Lett. {\bf 99}, 197401 (2007).

\item\emp{{\bf ** {Density functional study of X-ray Raman scattering from aromatic hydrocarbons and polyfluorene}}},\\ A. Sakko, M. Hakala, J. A. Soininen, and K. H\"am\"al\"ainen,  
Phys. Rev. B {\bf 76}, 205115 (2007).

\item{\emp{Development of a ReaxFF description for gold}},\\  T. T. J\"arvi, A. Kuronen, M. Hakala, K. Nordlund, A. C. T. van Duin, W. A. Goddard III, and T. Jacob,
Eur. Phys. J. B {\bf 66}, 75 (2008). 
% 75-79

\item\emp{{{Charge localization in alcohol isomers studied by Compton scattering}}},\\ M. Hakala, K. Nyg\aa rd, 
J. Vaara, M. Itou, Y. Sakurai and K. H\"am\"al\"ainen,
J.~Chem.~Phys. {\bf 130}, 034506 (2009). 

\item\emp{{\bf ** Structure of Liquid Linear Alcohols}},\\ J. S. Lehtola, M. Hakala and K. H\"am\"al\"ainen, J. Phys. Chem. B {\bf 114}, 6426 (2010).
% 6426-6436

\item\emp{Role of non-hydrogen-bonded molecules in the oxygen K-edge spectrum in ice},\\ T. Pylkk{\"a}nen, V. M. Giordano, J.-C. Chervin, A. Sakko, M. Hakala, J. A. Soininen, K. H\"am\"al\"ainen, G. Monaco and S. Huotari, J.\ Phys.\ Chem.\ B {\bf 114}, 3804 (2010).
% 3804-3808 

\item\emp{Amorphous defect clusters of pure Si and type inversion in Si detectors},\\ E. Holmstr{\"o}m, K. Nordlund and M. Hakala, Phys.\ Rev.\ B {\bf 82}, 104111 (2010).
% 5 pages

\item\emp{Anomalous Energetics in Tetrahydrofuran Clathrate Hydrate Revealed by X-ray Compton Scattering},\\ F. Lehmk{\"u}hler, A. Sakko, C. Sternemann,  M. Hakala, K. Nyg{\aa}rd, Ch. J. Sahle, S. Galambosi, I.
Steinke, S. Tiemeyer, A. Nyrow, T. Buslaps, D. Pontoni, M. Tolan, and K. H{\"a}m{\"a}l{\"a}inen, J. Phys. Chem. Lett. {\bf 1}, 2832 (2010).  
%pp 2832-2836  

\item\emp{Universal Signature of Hydrogen Bonding in the Oxygen K-Edge Spectrum of Alcohols},\\ T. Pylkk{\"a}nen, J. Lehtola, M. Hakala, A. Sakko, G. Monaco, S. Huotari, and K. H\"am\"al\"ainen, J. Phys. Chem. B {\bf 114}, 13076 (2010). 
%13076-13083

\item\emp{\bf ** {Time-dependent density functional approach for the calculation of inelastic x-ray scattering spectra of molecules}},\\ 
A. Sakko, A. Rubio, M. Hakala, and K. H{\"a}m{\"a}l{\"a}inen, J. Chem. Phys. {\bf 133}, 174111 (2010).  
% 6 pages

\item\emp{Nuclear magnetic resonance parameters in water dimer},\\ T. S. Pennanen, P. Lantto, M. Hakala and J. Vaara, Theor. Chem. Acc. {\bf 129}, 313 (2011). 
%DOI 10.1007/s00214-010-0782-y (2010).

\item\emp{Experimental and computational study of crystalline formic acid composed of the higher-energy cis conformer},\\ M. Hakala, K. Marushkevich, L. Khriachtchev, K. H\"am\"al\"ainen, and M. R\"as\"anen, J. Chem. Phys. {\bf 134}, 054506 (2011). %  

\item\emp{{{Calculation of isotropic Compton profiles with Gaussian basis sets}}},\\ J. Lehtola, M. Hakala, J. Vaara, and K. H\"am\"al\"ainen, Phys. Chem. Chem. Phys. {\bf 13}, 5630 (2011). % 5630 - 5641  

\item\emp{{{Inelastic x-ray scattering and vibrational effects at the $K$-edges of gaseous N$_2$, N$_2$O, and CO$_2$}}},\\ A. Sakko, S. Galambosi, J. Inkinen, T. Pylkk\"anen, M. Hakala, S. Huotari, and K. H\"am\"al\"ainen, Phys. Chem. Chem. Phys. {\bf 13}, 11678 (2011). % 11678-11685 

\item\emp{Reexamining the Lyman-Birge-Hopfield Band of N$_2$},\\ J. A. Bradley, A. Sakko, G. T. Seidler, A. Rubio, M. Hakala, K. H\"am\"al\"ainen, G. Cooper, A. P. Hitchcock, K. Schlimmer, and K. P. Nagle, Phys. Rev. A {\bf 84}, 022510 (2011). % 8 pages

\item\emp{Temperature Induced Structural Changes of Tetrahydrofuran Clathrate and of the Liquid Water/Tetrahydrofuran Mixture}, \\F. Lehmk\"uhler, A. Sakko, I. Steinke, C. Sternemann, M.
Hakala, C. J. Sahle, T. Buslaps, L. Simonelli, S. Galambosi, M. Paulus, T. Pylkkänen, M. Tolan, and K. H\"am\"al\"ainen, J. Phys. Chem. C {\bf 115}, 21009 (2011). 
%pp 21009–21015
%http://pubs.acs.org/doi/abs/10.1021/jp207027p
%Publication Date (Web): September 13, 2011 (Article)
%DOI: 10.1021/jp207027p

\item\emp{{\bf ** {Measurement of two solvation regimes in water-ethanol mixtures using x-ray Compton scattering}}},
\\ I. Juurinen, K. Nakahara, N. Ando, T. Nishiumi, H. Seta, N. Yoshida, T. Morinaga, M. Itou, T. Ninomiya, 
Y. Sakurai, E. Salonen, K. Nordlund, K. H{\"a}m{\"a}l{\"a}inen, and M. Hakala, Phys. Rev. Lett. {\bf 107}, 197401 (2011). 
%5 pages
%http://link.aps.org/doi/10.1103/PhysRevLett.107.197401
%DOI:10.1103/PhysRevLett.107.197401
%PACS:78.70.Ck, 33.15.Dj, 61.25.Em


\item\emp{Temperature dependence of the near-edge spectrum of water},\\
T. Pylkk{\"a}nen, A. Sakko, M. Hakala, K. H\"am\"al\"ainen, G. Monaco, 
and S. Huotari, J. Phys. Chem. B {\bf 115}, 14544 (2011).
%http://pubs.acs.org/doi/abs/10.1021/jp2015462
%J. Phys. Chem. B, Just Accepted
%Publication Date (Web): October 31, 2011 (Article)
%DOI: 10.1021/jp2015462
% ACKN 
% 1127462, 1256211, and 1254065
% Centers of Excellence Program (2006-2011)
% National Graduate School in Materials Physics
% Research Funds of the University of Helsinki (contract nos. 490064 and 490076)
% Christian Henriquet for expert technical assistance


%%%% 2012 %%%%

\item\emp{ERKALE - A Flexible Program Package for X-ray Properties of Atoms and Molecules},\\ J. Lehtola, M. Hakala, A. Sakko, and K. H\"am\"al\"ainen, J. Comput. Chem. {\bf 33}, 1572 (2012).
%1572-1585
%DOI: 10.1002/jcc.22987 {\it in press}.

\item\emp{Completeness-optimized basis sets: Application to ground-state electron momentum densities},\\ J. Lehtola, P. Manninen, M. Hakala, and K. H\"am\"al\"ainen, J. Chem. Phys. {\bf 137}, 104105 (2012).
% 8 pages



%%%% 2013 %%%%
\item\emp{Contraction of completeness-optimized basis sets: Application to ground-state electron momentum densities},\\ S. S. Lehtola, P. Manninen, M. Hakala, and K. H\"am\"al\"ainen, J. Chem. Phys. {\bf 138}, 044109 (2013).
% 8 pages
% http://dx.doi.org/10.1063/1.4788635


\item\emp{\bf ** {Microscopic structure of water at elevated pressures and temperatures}}, \\C. J. Sahle, C. Sternemann, C. Schmidt, S. S. Lehtola, S. Jahn, L. Simonelli, S. Huotari, M. Hakala, T. Pylkk{\"a}nen, A. Nyrow, K. Mende, M. Tolan, K. H{\"a}m{\"a}l{\"a}inen, and M. Wilke, Proc. Natl. Acad. Sci. USA {\bf 110}, 6301 (2013). 
% p. 6301-6303 3 p.
% http://dx.doi.org/10.1073/pnas.1220301110

\item\emp{Temperature dependence of CO2 and N2 core-electron excitation spectra at high pressure}, \\J. Inkinen, A. Sakko, K. O. Ruotsalainen, T. Pylkk{\"a}nen, J. Niskanen, S. Galambosi, M. Hakala, G. Monaco, S. Huotari, and K. H{\"a}m{\"a}l{\"a}inen, Phys. Chem. Chem. Phys. {\bf 15}, 9231 (2013) 
% p. 9231-9238 8 p.
% http://dx.doi.org/10.1039/C3CP50512J 
% Research Funds of the University of Helsinki (contracts No. 490064 and 490076) and the 
% Academy of Finland (contracts No. 1127462, 1254065, 1256211, 1260204, 
% and its Centres of Excellence Program 2012-2017)
% COMPX

\item\emp{Local changes of work function near rough features on Cu surfaces operated under high external electric field},\\
F. Djurabekova, A. Ruzibaev, E. Holmstr\"om, S. Parviainen, and M. Hakala, J. Appl. Phys. {\bf 114}, 243302 (2013).
% http://scitation.aip.org/content/aip/journal/jap/114/24/10.1063/1.4856875
% 490064

\item\emp{Saturation Behaviour in X-ray Raman Scattering Spectra of Aqueous LiCl},\\I. Juurinen, T. Pylkk\"anen, K. O. Ruotsalainen, C. Sahle, G. Monaco, K. H{\"a}m{\"a}l{\"a}inen, S. Huotari, and M. Hakala,  J. Phys. Chem. B {\bf 117}, 16506 (2013)
% 16506-16511 
% http://pubs.acs.org/doi/full/10.1021/jp409528r
% 1259526, 1256211, 490064, 490076


%%%% 2014 %%%%

\item\emp{Interplay between Temperature-Activated Vibrations and Nondipolar Effects in the Valence Excitations of the CO2 Molecule},\\ 
J. Inkinen, J. Niskanen, A. Sakko, K. O. Ruotsalainen, T. Pylkk{\"a}nen, S. Galambosi, M. Hakala, G. Monaco, K. H{\"a}m{\"a}l{\"a}inen, and S. Huotari, 
J. Phys. Chem. A {\bf 118}, 3288 (2014)
% 118 (18), pp 3288-3294
% 490064 and 490076, and the Academy of Finland (Contracts 1127462, 1254065, 1256211, 1260204, 
% and its Centres of Excellence Program 2012-2017)
% http://pubs.acs.org/doi/full/10.1021/jp5019058
% COMPX

\item\emp{Crystal-field excitations in NiO under high pressure studied by resonant inelastic x-ray scattering},\\
S. Huotari, L. Simonelli, V. M. Giordano, A. E. Rintala, Ch. J. Sahle, M. Hakala, P. Glatzel, R. Verbeni, and G. Monaco,
J. Phys.: Condens. Matter {\bf 26}, 135501 (2014)
% http://iopscience.iop.org/article/10.1088/0953-8984/26/13/135501/meta
% Academy of Finland (grants 1256211, 1254065, and 1260204) and University of Helsinki Research Funds (grant 490076) 
% COMPX


\item\emp{\bf ** {Molecular-Level Changes of Aqueous Poly(N-isopropylacrylamide) in Phase Transition}},\\
I. Juurinen, S. Galambosi, A. G. Anghelescu-Hakala, J. Koskelo, V. Honkim{\"a}ki, K. H{\"a}m{\"a}l{\"a}inen, S. Huotari, and M. Hakala, J. Phys. Chem. B {\bf 118}, 5518 (2014)
% 5518-5523
% Academy of Finland (NGSMP, contract numbers 1259526, 1256211, and 1254065), 
% the Research Funds of the University of Helsinki (projects 490064 and 490076), and Väisälä foundation
% http://pubs.acs.org/doi/full/10.1021/jp501913p
% 

\item\emp{Effect of the Hydrophobic Alcohol Chain Length on the Hydrogen-Bond Network of Water},\\
I. Juurinen, T. Pylkk{\"a}nen, Ch. J. Sahle, L. Simonelli, K. H{\"a}m{\"a}l{\"a}inen, S. Huotari, and M. Hakala, J. Phys. Chem. B {\bf 118}, 8750 (2014)
% 6 pages 
% 8750-8755
% Academy of Finland [NGSMP, Contract Nos. 1259526, 1256211, 1254065, 1259599, and 1260204] and 
% the Research Funds of the University of Helsinki (Projects 490064 and 490076)
% http://pubs.acs.org/doi/full/10.1021/jp5045332
% COMPX, QUASIM

\item\emp{\bf ** {Multi-intermediate-band character of Ti-substituted CuGaS2: 
Implications for photovoltaic applications}},\\ 
J. Hashemi, A. Akbari, S. Huotari, and M. Hakala, Phys. Rev. B {\bf 90}, 075154 (2014) % 5 pages 
% Academy of Finland (Contracts No. 1260204, No. 1259599, No. 1259526, and No. 1256211)
% http://journals.aps.org/prb/abstract/10.1103/PhysRevB.90.075154
% COMPX, QUASIM

\item\emp{Intra- and intermolecular effects on the Compton profile of the ionic liquid 1,3-dimet\-hylimidazolium chloride},\\ 
J. Koskelo, I. Juurinen, K. O. Ruotsalainen, M. McGrath, I.-F. Kuo, S. Lehtola, S. Galambosi, K. H{\"a}m{\"a}l{\"a}inen, S. Huotari and M. Hakala, The Journal of Chemical Physics {\bf 141}, 244505 (2014) 
% 6 pages
% J. Chem. Phys. 141, 244505 (2014)
% http://dx.doi.org/10.1063/1.4904278 
% Academy of Finland (NGSMP, Contract Nos. 1259526, 1256211, 1254065, 1259599, 1260204, and 1283136), 
% the Research Funds of the University of Helsinki (Project Nos. 490064 and 490076), and Väisälä foundation. 
% COMPX, QUASIM

%%%% 2015 %%%%

\item\emp{Identification of the dye adsorption modes in dye-sensitised solar cells with X-ray spectroscopy techniques: a computational study},\\
A. Akbari, J. Hashemi, J. Niskanen, S. Huotari, and M. Hakala, Phys. Chem. Chem. Phys. {\bf 17}, 10849 (2015) 
% 7 pages
% 10849-10855
% http://pubs.rsc.org/en/content/articlelanding/2015/cp/c4cp05980h#!divAbstract
% Academy of Finland (contract numbers 1259599, 1260204, 1254065, 1283136, and 1259526)
% COMPX, QUASIM

\item\emp{Inelastic x-ray scattering in heterostructures: electronic excitations in LaAlO3/SrTiO3},\\
K. O. Ruotsalainen, C. J. Sahle, T. Ritschel, J. Geck, M. Hosoda, C. Bell, Y. Hikita, H. Y. Hwang, T. T. Fister, R. A. Gordon, K. H{\"a}m{\"a}l{\"a}inen, M. Hakala, and S. Huotari, J. Phys.: Condens. Matter {\bf 27}, 335501 (2015)
% http://iopscience.iop.org/article/10.1088/0953-8984/27/33/335501/meta
% Academy of Finland (Grants 1260204, 1256211, 1127462, 1259526 and 1254065) and University of Helsinki Research Funds.
% COMPX

\item\emp{Exciton energy-momentum map of hexagonal boron nitride},\\
G. Fugallo, M. Aramini, J. Koskelo, K. Watanabe, T. Taniguchi, M. Hakala, S. Huotari, M. Gatti, and F. Sottile, Phys. Rev. B {\bf 92}, 165122 (2015)
% Academy of Finland (Contracts No. 1259599, No. 1260204, No. 1254065, No. 1283136, and No. 1259526)
% COMPX, QUASIM
% http://journals.aps.org/prb/abstract/10.1103/PhysRevB.92.165122

\item\emp{Protonation Dynamics and Hydrogen Bonding in Aqueous Sulfuric Acid},\\
J. Niskanen, C.J. Sahle, I. Juurinen, J. Koskelo, S. Lehtola, R. Verbeni, H. M{\"u}ller, M. Hakala, and S. Huotari, J. Phys. Chem. B {\bf 119}, 11732 (2015)
% Academy of Finland is acknowledged for funding through the projects 1260204 and 1259599
% COMPX, QUASIM
% http://pubs.acs.org/doi/full/10.1021/acs.jpcb.5b04371

\item\emp{X-ray induced dimerization of cinnamic acid: Time-resolved inelastic X-ray scattering study},\\ 
J. Inkinen, J. Niskanen, T. Talka, C. Sahle, H. M{\"u}ller, L. Khriachchev, J. Hashemi, A. Akbari, 
M. Hakala and S. Huotari, Sci. Rep. {\bf 5}, 15851 (2015)
% Academy of Finland projects 1259526, 1283136, 1254065, 1260204, 1259599, and 1277993 
% as well as by Vilho, Yrjö, and Kalle Väisälä Foundation
% COMPX, QUASIM
% http://www.nature.com/articles/srep15851


%%%% 2016 %%%%

\item\emp{Probing the thermal stability and decomposition mechanism of
    a magnesium-fullerene polymer via X-ray Raman spectroscopy, X-ray
    diffraction and molecular
    dynamics simulations},\\
M. Aramini, J. Niskanen, C. Cavallari, D. Pontiroli, A. Musazay, M. Krisch, M. Hakala and S. Huotari, Phys. Chem. Chem. Phys. {\bf 18}, 5366 (2016)
% http://pubs.rsc.org/en/content/articlelanding/2016/cp/c5cp07783d#!divAbstract
% Academy of Finland (grants No. 1254065, 1283136, 1259526, 1259599 and 1260204)
% COMPX, QUASIM

\item\emp{Sulphur Kb emission spectra reveal protonation states of aqueous sulfuric acid},\\
J. Niskanen, C.J. Sahle, K. O. Ruotsalainen, H. M{\"u}ller, M. Kavcic, M. Zitnik, K. Bucar, M. Petric, M. Hakala and S. Huotari, Sci. Rep. {\bf 6}, 21012 (2016)
% http://www.nature.com/articles/srep21012
% 124065, 1283136, 1259526, 1260204 and 1259599
% COMPX, QUASIM

\item\emp{Resonant X-ray emission with a standing wave excitation},\\
K.O. Ruotsalainen, A.-P. Honkanen, S.P. Collins, G. Monaco, Moretti M. Sala, M. Krisch, K. H{\"a}m{\"a}l{\"a}inen, M. Hakala and S. Huotari, {\bf 6}, 22648 (2016)
% 1254065, 1283136, 1259526, 1260204 and 1259599
% http://www.nature.com/articles/srep22648
% COMPX, QUASIM

\item\emp{Intramolecular structure and energetics in supercooled water down to 255 K},\\
F. Lehmk{\"u}hler, Y. Forov, T. Büning, C.J. Sahle, I. Steinke, K. Julius, T. Buslaps, M. Tolan, M. Hakala and C. Sternemann, Phys. Chem. Chem. Phys. {\bf 18}, 6925 (2016)
% 1260204 and 1259599
% http://pubs.rsc.org/en/content/articlehtml/2016/cp/c5cp07721d
% COMPX, QUASIM
\href{http://pubs.rsc.org/en/content/articlehtml/2016/cp/c5cp07721d}{www}

\item\emp{First-principles analysis of the intermediate band in CuGa(1-x)FexS2},\\
J. Koskelo, J. Hashemi, S. Huotari and M. Hakala, Phys. Rev. B {\bf 93}, 165204 (2016)
% MATRENA Doctoral Programme, 
% Academy of Finland (Contracts No. 1256211, No. 1254065, No. 1259599, No. 1260204, No. 1259526, and No. 1283136), 
% and Väisälä foundation
% http://journals.aps.org/prb/abstract/10.1103/PhysRevB.93.165204
% COMPX, QUASIM


\item\emp{First principles modeling of perovskite solar cells based on TiO$_2$ and Al$_2$O$_3$: Stability and Interfacial Electronic Structure},\\
A. Akbari, J. Hashemi, E. Mosconi, F. De Angelis and M. Hakala, submitted to J. Phys. Chem. Lett. (2016) 
% Academy of Finland (No. 1259599, No. 1260204)
% COMPX, QUASIM, OptPero
 




%%%% Submitted %%%%

%%%\hspace{3cm} - - - - - - - - - - - - - - - - - - - - - - - - - -



%%%% In preparation %%%%

%\hspace{3cm} - - - - - - - - - - - - - - - - - - - - - - - - - -



%\newpage



\end{enumerate}

\vspace{0.4cm}

%%%\noindent B2 Artikkelit kokoomateoksissa ja kongressijulkaisuissa
%%\noindent {\bf Conference Papers}
%%%\noindent {\bf CONFERENCE PAPERS}
%\noindent {\bf Artikkelit kokoomateoksissa}
%\noindent {\bf Conference papers}
\noindent {\bf Non-refereed scientific articles}

\begin{enumerate}

\item \emp{First-Principles Calculations of Positron Annihilation in
Solids},\\B. Barbiellini, M. Hakala, R. M. Nieminen, and M. J. Puska,
Proceedings of the MRS Fall Meeting, Boston, USA, 1999.

\end{enumerate}



%\noindent {\bf Publications intended for professional communities}
%
%
%


\vspace{0.4cm}

%%%\noindent E Muut julkaisut
%%%\noindent {\bf OTHER PUBLICATIONS}
%\noindent {\bf Muut julkaisut}
%\noindent {\bf Other publications}
%\noindent {\bf Publications intended for the general public}
\noindent {\bf Publications intended for the general public}




\begin{enumerate}

%\item \emp{Synkrotronis\"ateily paljastaa aineen rakenteen}, K.~H\"am\"al\"ainen ja M.~Hakala, radiohaastattelu, YLE, 25.1.2006.
\item \emp{Synkrotronis\"ateily paljastaa aineen rakenteen}, K.~H\"am\"al\"ainen and M.~Hakala, Radio interview (in Finnish), Finnish Broadcasting Company (YLE), 25.1.2006

%\item \emp{Approach to Cold Heat-Storage Mechanism of Ice}, K.~H\"am\"al\"ainen, S.~Manninen, K.~Nyg{\aa}rd, M.~Hakala, M.~Itou and Y. Sakurai, lehdist\"otiedote, SPring-8, Japan, 8.11.2007.
\item \emp{Approach to Cold Heat-Storage Mechanism of Ice}, K.~H\"am\"al\"ainen, S.~Manninen, K.~Ny\-g{\aa}rd, M.~Hakala, M.~Itou and Y. Sakurai, Press release, SPring-8, Japan, 8.11.2007.

%\item \emp{Uutta tietoa veden l\"amp\"oominaisuuksista r\"ontgensironnalla}, lehdist\"otiedote, STT, 14.11.2007.
\item \emp{Uutta tietoa veden l\"amp\"oominaisuuksista r\"ontgensironnalla}, Press release (in Finnish), The Finnish News Agency (STT), 14.11.2007.

\item \emp{Configurational energetics in ice Ih probed by Compton scattering}, K. Nyg{\aa}rd, M. Hakala, and K. H\"am\"al\"ainen, SPring-8 Research Frontiers 2007, Japan.

\item \emp{New information on thermal properties of water through
X-ray scattering technique}, CSC News 1/2008, p.~9.

\item \emp{Nestem\"aisten lineaaristen alkoholien rakenneanalyysi}, CSC Ajankohtaista (in Finnish), 17.5.2010.

%\item \emp{Kohti mahdollisimman tarkkoja malleja}, haastattelu, CSC News 1/2011
\item \emp{Striving for the best possible accuracy in models}, interview, CSC News 1/2011, p.~4. 

\item \emp{Molekyylitason rakennetutkimusta r{\"o}ntgenmenetelmin}, M. Hakala, Arkhimedes {\bf 1}, 14 (2012).
% pp. 14-20, 6 pages

\item \emp{Ethanol-water structures at the microscopic level studied by X-ray Compton scattering: extreme sensitivity to geometries}, M. Hakala, I. Juurinen and K. Nakahara, SPring-8 Research Frontiers 2011, Japan.

\item \emp{Scientists probe atomic structure and dynamics of water under deep Earth extreme pressure and temperature conditions}, C. J. Sahle, C. Sternemann, C. Schmidt, S. S. Lehtola, S. Jahn, L. Simonelli, S. Huotari, M. Hakala, T. Pylkk{\"a}nen, A. Nyrow, K. Mende, M. Tolan, K. H{\"a}m{\"a}l{\"a}inen, and M. Wilke, ESRF News 8.3.2013

\item \emp{Microscopic structure of water under conditions of the Earth's crust and mantle}, C. J. Sahle, C. Sternemann, C. Schmidt, S. S. Lehtola, S. Jahn, L. Simonelli, S. Huotari, M. Hakala, T. Pylkk{\"a}nen, A. Nyrow, K. Mende, M. Tolan, K. H{\"a}m{\"a}l{\"a}inen, and M. Wilke, ESRF Highlights 2013

\end{enumerate}



\newpage
%\vspace{0.4cm}


\noindent {\bf Theses}

\begin{enumerate}

\item Master's thesis: \emp{Computational Scheme for Core-Electron Annihilation in Solids}, Hel\-sin\-ki University of Technology (1996)

\item Doctoral dissertation: \emp{Defect Complexes in Silicon: Electronic Structures and Positron Annihilation}, Helsinki University of Technology (2001) 


\end{enumerate}

%\newpage

%\end{document}

%\newpage

% THE FOLLOWING SECTION NOT OFFICIALLY IN THE ACADEMY'S GUIDELINE
%\noindent {\bf Kutsutut puheet}
%\noindent {\bf Other activity: Invited talks}
\noindent {\bf Invited talks}

\begin{enumerate}

\item \emp{X-ray Compton scattering as a probe of hydrogen bonds and local coordination in water}\\
Stockholm Discussion Meeting, 14.-16.6.2006, Albanova University Center, Stockholm University, Sweden   %Ruotsi

\item \emp{Hydrogen Bonds in Water and Aqueous Systems Studied by Compton Scattering and DFT Calculations}\\
Sagamore XV Conference, 13.-18.8.2006, University of Warwick, Warwickshire, United Kingdom              %Iso-Britannia

\item \emp{What Compton Scattering Tells about the Intra- and Intermolecular structure of Water}\\
Bunsen-Kolloquium: Chemical Bonding in Position, Momentum, and Phase Space, 5.-6.2.2007, Univ. Konstanz, Germany     %Saksa

\item \emp{Compton scattering as a probe of hydrogen bonds and molecular structure of aqueous systems}\\
6th International Conference on Inelastic X-ray Scattering, 7.-11.5.2007, Awaji, Japan

\item \emp{Liquids and molecular systems by X-ray Compton and Raman scattering}\\ 
5th Summer School for Synchrotron Radiation Users, 11.-13.8.2008, Kuortane, Finland

\item \emp{Electronic properties of molecular structures by inelastic X-ray scattering}\\ 
XIV International Workshop on Quantum Systems in Chemistry and Physics, 13.-19.9.2009, Madrid, Spain

\item \emp{Sub-nanometer properties of materials by theoretical and experimental Compton scattering}\\
Sagamore XVII Conference, 15.-20.7.2012, Kitayuzawa, Hokkai-do, Japan

\item \emp{Electronic structure and x-ray properties}\\
Summer School on Novel Approaches to Electronic Structure Theory, 15.-17.8.2012, Tampere, Finland

\item \emp{Structure of water studied by inelastic x-ray scattering}\\
7th International Discussion Meeting on Relaxations in Complex Systems, 21.-26.7.2013, Barcelona, Spain

\item \emp{Non-resonant inelastic x-ray scattering in molecular systems: 
sensitivity to geometries}\\
8th International Conference on Inelastic X-ray Scattering, 11.-16.8.2013, Menlo Park, CA, USA

\item \emp{Inelastic x-ray scattering spectroscopy}\\
Winter School in Theoretical Chemistry 2013: Theoretical Spectroscopy, 18.12.2013, Helsinki, Finland

\item \emp{Social media - Possibilies to popularize history}\\
Helsinki Summer University, 21.6.2016, Helsinki, Finland
% https://courses.helsinki.fi/a406021/113041266

%\item \emp{Sosiaalinen media - Mahdollisuuksia historian popularisointiin}\\
%Helsingin seudun kes{\"a}yliopisto, 21.6.2016, Helsinki, Finland


\end{enumerate}


\end{document}

\newpage

\noindent {\bf MIKKO HAKALA} \\

\noindent {\bf TEN MOST IMPORTANT PUBLICATIONS} \hfill 29.5.2012\\

\begin{enumerate}


\item \emp{{\bf {Theoretical and experimental study of positron annihilation
with core electrons in solids}}},\\M. Alatalo, B. Barbiellini,
M. Hakala, H. Kauppinen, T. Korhonen, M. J. Puska, K. Saarinen,
P. Hautoj\"arvi, and R. M. Nieminen, Phys. Rev. B. {\bf 54}, 2397
(1996).


This publication was my first scientific paper and belongs to the most
influential ones (over 150 citations).  I carried out the
programming, calculations and other data handling, and wrote parts of
the paper. The interpretations of the results were done by all the
authors. In addition, I combined the various computational aspects to
a software package. The paper has made a strong impact on the whole
field of positron spectroscopy of solid state. It offered a method to
quantitatively compare experimental data to computations, and the
method is in continuous use.

%The positron group in HUT, led by Filip Tuomisto (successor of late
%Prof. Kimmo Saarinen) has been very influential worldwide and I am
%proud to have contributed in the late 90's to that group's
%achievements. 



\item \emp{{\bf {Momentum distributions of electron-positron pairs
 annihilating at vacancy clusters in Si}}}, \\M. Hakala, M. J. Puska,
 and R. M. Nieminen, Phys. Rev. B. {\bf 57}, 7621 (1998).

This work was carried out by me under the supervision of M. Puska and R. M. Nieminen. It
continues and builds on the research described above, by including a
realistic modelling of momentum distribution from valence electrons of
a crystalline solid. I did all the calculations and analysis of the
results and wrote the paper. It is well cited (over 90 citations) and
serves as a continuous reference to various groups performing positron
annihilation experiments worldwide.


\item \emp{{{\bf {First-principles calculations of interstitial boron in
silicon}}}},\\ M. Hakala, M. J. Puska, and R. M. Nieminen, Phys. Rev. B
{\bf 61}, 8155 (2000).

First-principles prediction of the electronic properties of materials
is one of the cornerstones of modern materials research. This
computational research on boron impurities in bulk silicon was carried
out by me, including planning, computations, analysis and
interpretation. I used plane-wave pseudopotential density-functional
theory calculations and learned thoroughly the method for
state-of-the-art lattice defect calculations. The study has made an
important impact (over 60 citations).



\item\emp{{\bf {Compton profiles for water and mixed water-neon clusters: A
measure of coordination}}},\\ M. Hakala, S. Huotari,
K. H\"am\"al\"ainen, S. Manninen, Ph. Wernet, A. Nilsson, and
L. G. M. Pettersson, Phys. Rev. B {\bf 70}, 125413 (2004).

This work is the first publication resulting from my new research
interests after moving to the x-ray laboratory in University of
Helsinki, where I started as a post-doc resercher in 2003. I planned
this computational work in collaboration with the researchers in
Stocholm university (Profs.~Lars Pettersson and Anders Nilsson, who
are well-known for their work on x-ray core-level spectroscopies and
liquids.)  The work involved programming of an x-ray Compton
scattering module to an existing software package StoBe-deMon, which
is a widely used software for x-ray properties, and performed a set of
model calculations. The code is now in a routine use in our laboratory
for interpreting Compton scattering spectra in molecular systems.  In
brief, the paper is a seminal work on how to do calculations to
interpret and predict experimental synchrotron data on molecular
systems. This paper was the first in the field of water research in
University of Helsinki, where I was
the driving force for both experiments and computations.


%\item\emp{{{{\bf {Intra- and intermolecular effects in the Compton profile of
%water}}}}},\\ M. Hakala, K. Nyg\aa rd, S. Manninen, L. G. M. Pettersson
%and K. H\"am\"al\"ainen, 
%Phys. Rev. B {\bf 73} 035432 (2006).

%This project was fully designed by me. I also did most of the
%calculations and wrote most of the paper. The paper documents the
%essential computational findings for predicting and understanding
%properties measured in Compton experiments, and it serves therefore as 
%an important reference.


\item\emp{{\bf {Gold as intermolecular glue: a predicted planar
triaurotriazine, C$_3$Au$_3$N$_3$, isomer of gold cyanide}}},\\
M. O. Hakala and P. Pyykk\"o, Chem. Commun., 2890 (2006). 

This work was initiated by Prof.~Pekka Pyykk\"o. He gave the initial
idea for predicting novel periodic molecular compounds, not yet seen
experimentally. I planned and did all the computations. The
intepretation and conclusions were written together. The work is one
of the outcomes of the Centre of Excellence in Computation Molecular
Science (of Academy of Finland), which was operational 2006-2011 and to
which I belonged. The work illustrates the importance of networking and
extending into new areas.


%\item\emp{{{\bf {Correlation of hydrogen bond lengths and angles in liquid
%water based on Compton scattering}}}},\\ M. Hakala, K. Nyg\aa rd,
%S. Manninen, S. Huotari, T. Buslaps, A. Nilsson, L. G. M. Pettersson,
%and K. H\"am\"al\"ainen,  J. Chem. Phys. {\bf 125}, 084504 (2006). % -1-7


\item\emp{{\bf {Configurational energetics in ice Ih probed by Compton scattering}}},\\ 
K. Nyg\aa rd, M. Hakala, S. Manninen, M. Itou, Y. Sakurai and K. H\"am\"al\"ainen,  
Phys. Rev. Lett. {\bf 99}, 197401 (2007).

This paper is one of the highlights of the various water research
projects that I carried out with the collaborators in University of Helsinki during
2003-2007. I carried out together with Kim Nyg{\aa}rd the physical
interpretation of the experimental results. The paper has turned out
to be an important opening in terms of energetics studied by Compton
scattering.


\item\emp{{\bf {Density functional study of X-ray Raman scattering from aromatic hydrocarbons and polyfluorene}}},\\ A. Sakko, M. Hakala, J. A. Soininen, and K. H\"am\"al\"ainen,  
Phys. Rev. B {\bf 76}, 205115 (2007).

In this work we developed a new computational method to intepret x-ray
Raman scattering spectra of molecular systems. The original
computational ideas were initiated by me, while programming,
calculations, writing and further developements were carried out
mainly be A. Sakko. This work represents a logical continuation of my
efforts to develop computational methodologies that can be used to
interpret and predict experimentally measured quantities. These
methods are now in use and have direct relevance to planning and
interpreting experimental data measured at synchrotron laboratories.


%\item\emp{\bf {Structure of Liquid Linear Alcohols}},\\ J. S. Lehtola, M. Hakala and K. H\"am\"al\"ainen, J. Phys. Chem. B {\bf 114}, 6426 (2010).
%% 6426-6436

%In this paper pure common alcohols were comprehensively studied by
%molecular dynamics simulations. The project was suggested and planned
%by me. The practical realization, computations and the further
%analysis was done by PhD student J. Lehtola. The paper has already
%made an impact and is cited well.


\item\emp{\bf {Time-dependent density functional approach for the calculation of inelastic x-ray scattering spectra of molecules}},\\ 
A. Sakko, A. Rubio, M. Hakala, and K. H{\"a}m{\"a}l{\"a}inen, J. Chem. Phys. {\bf 133}, 174111 (2010).  
% 6 pages

The time-propagation scheme and Casida's method of the time-dependent
density functional theory are extended to calculate the full
wavevector dependent response function. The method provides a novel
analysis tool for spectroscopic methods such as inelastic x-ray
scattering and electron energy loss spectroscopy. I contributed to the
brainstorming and planning of the paper and to the interpretation of
the computational data.



\item\emp{{\bf {Calculation of isotropic Compton profiles with Gaussian basis sets}}},\\ J. Lehtola, M. Hakala, J. Vaara, and K. H\"am\"al\"ainen, Phys. Chem. Chem. Phys. {\bf 13}, 5630 (2011). % 5630 - 5641  

In this paper significant advances towards chemical accuracy and new
methods in calculating inelastic x-ray scattering in the Compton
regime for molecules was done. The project was suggested and planned
by me, while the practical realization, computations and the further
analysis was done by PhD student J. Lehtola.

%{\bf PROBLEM-ORIENTED RESEARCH}\\

\item\emp{{\bf {Measurement of two solvation regimes in water-ethanol mixtures using x-ray Compton scattering}}},\\ I. Juurinen, K. Nakahara, N. Ando, T. Nishiumi, H. Seta, N. Yoshida, T. Morinaga, M. Itou, T. Ninomiya, Y. Sakurai, E. Salonen, K. Nordlund, K. H{\"a}m{\"a}l{\"a}inen, and M. Hakala, Phys. Rev. Lett. {\bf 107}, 197401 (2011). 

By combining synchrotron experiments, molecular dynamics simulations
and spectra calculations, this work characterizes the subtle
microscopic structure of water-ethanol mixtures as a function of
concentration. The experimental Compton scattering data was obtained
by a collaborating group in Japan, which contacted us for
intepretation. I planned the ensuing project, the structure of the
paper, its main hypotheses, and the computational work. The
computations and analysis were done by PhD student Iina Juurinen and
partly by myself.


\end{enumerate}



\end{document}









\noindent {\bf Ten most important publications relevant to the research plan}\\

\noindent {\it a) Molecular systems studied by X-ray methods}

\begin{itemize}

\item\emp{Correlation of hydrogen bond lengths and angles in liquid
water based on Compton scattering},\\ M. Hakala, K. Nyg\aa rd,
S. Manninen, S. Huotari, T. Buslaps, A. Nilsson, L. G. M. Pettersson,
and K. H\"am\"al\"ainen,  J. Chem. Phys. {\bf 125}, 084504 (2006). % -1-7

%\item\emp{Compton scattering study of water versus ice Ih: Intra- and intermolecular structure},\\ K. Nyg\aa rd, M. Hakala, S. Manninen,
%A. Andrejczuk, M. Itou, Y. Sakurai, L. G. M. Pettersson
%and K. H\"am\"al\"ainen,  
%Phys. Rev. E {\bf 74}, 031503 (2006).

\item\emp{Electronic stucture of methane hydrate studied by Compton scattering},\\ C. Sternemann,
S. Huotari, M. Hakala, M. Paulus,
M. Volmer, C. Gutt, T. Buslaps, N. Hiraoka, D. D. Klug, 
K. H{\"a}m{\"a}l{\"a}inen, M. Tolan, and J. S. Tse, 
Phys. Rev. B {\bf 73}, 195104 (2006). % -1-6

\item\emp{Isotope quantum effects in the Compton profile of water},\\ K. Nyg\aa rd, M. Hakala, T. Pylkk{\"a}nen, S. Mannine, T. Buslaps, M. Itou, 
A. Andrejczuk, Y. Sakurai, M. Odelius and K. H\"am\"al\"ainen,  
J. Chem. Phys. {\bf 126}, 154508 (2007).

\item\emp{Configurational energetics in ice Ih probed by Compton scattering},\\ K. Nyg\aa rd, M. Hakala, S. Manninen,
M. Itou, Y. Sakurai and K. H\"am\"al\"ainen,  
Phys. Rev. Lett. {\bf 99}, 197401 (2007).

\item\emp{Charge localization in alcohol isomers studied by Compton scattering},\\ M. Hakala, K. Nyg\aa rd, 
J. Vaara, M. Itou, Y. Sakurai and K. H\"am\"al\"ainen,
J.~Chem.~Phys. {\bf 130}, 034506 (2009). 




\end{itemize}

\noindent {\it b) Methodological developments for X-ray studies}

\begin{itemize}

\item\emp{Calculation of valence electron momentum densities using the
projector\\augmented-wave method},\\ I. Makkonen, M. Hakala, and
M. J. Puska, J. Phys. Chem. Solids {\bf 66}, 1128 (2005).

\item\emp{Intra- and intermolecular effects in the Compton profile of
water},\\ M. Hakala, K. Nyg\aa rd, S. Manninen, L. G. M. Pettersson
and K. H\"am\"al\"ainen, 
Phys. Rev. B {\bf 73} 035432 (2006).

\item\emp{Density functional study of X-ray Raman scattering from aromatic hydrocarbons and polyfluorene},\\ A. Sakko, M. Hakala, J. A. Soininen, and K. H\"am\"al\"ainen,  
Phys. Rev. B {\bf 76}, 205115 (2007). 

\end{itemize}

\noindent {\it c) Prediction and properties of point defects and new structures}

\begin{itemize}

\item \emp{First-principles calculations of interstitial boron in
silicon},\\ M. Hakala, M. J. Puska, and R. M. Nieminen, Phys. Rev. B
{\bf 61}, 8155 (2000).

%\item\emp{Gold as intermolecular glue: a predicted planar
%triaurotriazine, C$_3$Au$_3$N$_3$, isomer of gold cyanide},\\
%M. O. Hakala and P. Pyykk\"o, Chem. Commun., 2890 (2006). 

\item\emp{Comparison of chain versus sheet crystal structures for the cyanides MCN (M=Cu-Au) and dicarbides 
MC$_2$ (M=Be-Ba,Zn-Hg)},\\P. Zaleski-Ejgierd, M. Hakala, and P. Pyykk\"o, 
Phys.~Rev.~B {\bf 76}, 094104 (2007). % -1-9



\end{itemize}



\end{document}
